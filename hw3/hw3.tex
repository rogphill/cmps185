\documentclass[a4paper,11pt]{article}
\usepackage[cm]{fullpage}
\setlength{\parindent}{0em}

\begin{document}

\begin{center}
    \huge Homework Three
\end{center}

\begin{center}
    Rob Phillips
\end{center}

\section*{Part I}

\subsection*{Selected ACM Journal}

\textbf{Name of publication:} ACM Transactions on Cyber-Physical Systems

\textbf{URL:} https://dlnext.acm.org/journal/tcps

\textbf{Publisher:} Association for Computing Machinery

\textbf{Name of the Editor-in-Chief:} Tei-Wei Kuo

\textbf{Size of the Editorial Board:} 27

\textbf{How often the journal is published:} Quarterly

\textbf{Page limit:} 25 pages

\textbf{Double-blind reviewing policy:} Single blind. Article submitter does not know the identity of the reviewer, but the reviewer knows the identity of the submitter.

\textbf{Open access policy:} Have to purchase each journal or subscribe.

\textbf{Additional interesting information:} I chose this journal because cyber-physical systems seemed like a really interesting topic. The journal has been around for a long time, and it seems that it is still extremely relevant with things like self-driving cars becoming incrasingly popular. 

\subsection*{Selected Elsevier Journal}

\textbf{Name of publication:} Journal of Logical and Algebraic Methods in Programming

\textbf{URL:} https://www.journals.elsevier.com/journal-of-logical-and-algebraic-methods-in-programming

\textbf{Publisher:} Elsevier

\textbf{Name of the Editor-in-Chief:} Rocco De Nicola

\textbf{Size of the Editorial Board:} 21

\textbf{How often the journal is published:} Bimonthly; every two months

\textbf{Page limit:} No listed page limit. Some articles are around 10 pages, with other articles having around 50 pages.

\textbf{Double-blind reviewing policy:} Single blind. Article submitter does not know the identity of the reviewer, but the reviewer knows the identity of the submitter.

\textbf{Open access policy:} Up to the researcher. Can select a subscription option, in which people subscribe to the journal, or a "gold open access" option, where the publication is available to everybody with the catch that the author needs to cover the publication fee.

\textbf{Additional interesting information:} Since I'm also a math major, I've always found the mathematical side of computer science to be really fascinating, and just how effortless these two disciplines blend together. This journal piqued my interest as a result. There are some really cool papers in their "open access" section that I was able to look at, as well.

\subsection*{Selected Springer Journal}

\textbf{Name of publication:} Journal of Cryptology

\textbf{URL:} https://link.springer.com/journal/145

\textbf{Publisher:} Springer US

\textbf{Name of the Editor-in-Chief:} Kenneth G. Paterson

\textbf{Size of the Editorial Board:} 27

\textbf{How often the journal is published:} Quarterly

\textbf{Page limit:} No listed page limit. Some papers are nearing 100 pages long.

\textbf{Double-blind reviewing policy:} Allows authors to follow a double-blind reviewing procuedure, but only if the submitted paper is not extended research continued by the same authors, so as to fairly evaluate the work.

\textbf{Open access policy:} Called "open choice." Springer will release the paper to the public in exchange for covering the publication charge.

\textbf{Additional interesting information:} Cryptology was historically and still is an important part of the field of computer science. This journal was actually really interesting to just browse through their open access articles.


\section*{Part II}\n

I am very much in favor of open access to research journals, which is where research papers are distributed freely online. I fundamentally do not think that knowledge should be restricted to just the elites, or to those with the most money. It feels that, in a sense, there's a huge wealth of knowledge that's gated behind a paywall, and it's hard for even a university student like ourselves to access it. As I was looking through journals for part I of this assignment, it was difficult to even access a lot of the papers to check them out. If that's the case, then how can people in the third world, or even economically disadvantaged people in our own country, access that knowledge? It feels morally and even economically wrong to me -- as, in my opinion, an educated populace is a healthy one and can only benefit an economy.\\

While I recognize that there still needs to be a source of money -- I still believe other options can be explored. I think that the open access models that Elsevier and Springer have are great starts. Those two publishers let the researcher find a way to cover the costs of publishing so that they can publish it with open access. However, I would love to see a world where information is free, period, and the university isn't offloaded with the cost which in turn affects student tuition. I would also be very interested to see some studies into "pay what you think it's worth" crowdfunding models, as this model seems like it works well in other industries. I would even like to see a governmental system in which we prioritize education, including free university and subsidized open access research papers.\\


\end{document}
